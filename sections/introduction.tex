\section{Introduction}\label{sec:introduction}

Most contemporary deployments of DNS-over-HTTPS (DoH)~\cite{rfc8484}. have occurred in browsers that provide limited options for resolvers~\cite{chromeResolvers,ffChoices}. Although DoH protects against on-path eavesdropping, it does not prevent resolvers themselves from seeing the contents of DNS queries. Thus, some have argued that browser-based DoH deployments shift privacy concerns from eavesdroppers to potential misuse by major DNS providers~\cite{vixie}.

Cloudflare, Cloudflare, Quad9, NextDNS, CleanBrowsing, and OpenDNS are the DoH resolvers that have been deployed
to users of major browsers as of April 14,
2025.  We
define these resolvers as {\em
mainstream}.
Yet, many other DoH resolvers have been deployed that are currently
not in use by major browser deployments~\cite{dnscrypt}.  

Importantly, while some large mainstream resolvers (e.g., Google, Cloudflare, Quad9, Hurricane Electric) are anycasted, many smaller and non-mainstream resolvers are not. Thus, our use of geolocation information is not intended as a perfect ground truth of server placement, but rather as a way to group resolvers and provide a more comprehensive set of measurements across vantage points.

We make the following contributions: (1) We measure DoH response times from a large list of resolvers, including both mainstream DoH resolvers that are included in major browser vendors and a large collection of non-mainstream resolvers. (2) We study how the performance of various DoH resolvers differ based on vantage point. (3) We study the performance of DoH resolvers in home networks. 
