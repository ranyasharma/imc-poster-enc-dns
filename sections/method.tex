\section{Methods}\label{sec:method}

We perform measurements across 74 DoH resolvers, grouped by their geographical locations—18 in North America, 13 in Asia, 5 in Australia, and 34 in Europe. 5 resolvers were unable to return a location. These resolvers were scraped from a list of public DoH resolvers provided by the DNSCrypt protocol developers. We employed MaxMind’s GeoLite2 databases to geolocate each DoH resolver. We also took the four highest performing resolvers (Google, Cloudflare, Quad9, Hurricane Electric) located in North America and measured their performance in Europe and Asia to better understand how they compare in farther vantage points. We issued queries for three domains to each resolver: google.com, amazon.com, and wikipedia.com. Our selection of domains is quite realistic, because it is reasonable to expect that most people query sites that are already in cache. 

We define DNS query response time as the
end-to-end time it takes for a client to initiate a query and receive a
response.  To measure query response times with various DoH resolvers, we
performed dig queries to the resolvers. We used the public, open-source Netrics
platform to conduct periodic measurements.

We performed network latency measurements for each recursive resolver.  Each time
we issued a set of DoH queries to a resolver, we also issued a ICMP ping
message and noted the round-trip time.

We collect continuous measurements via two sources—-home network devices and Amazon EC2 instances. The home network measurements were across four units in the same apartment complex in the Chicagoland area, collected between June 22--September 30, 2023. 
We installed our measurement platform on Raspberry Pi devices.
For the EC2 measurements, we deployed one server
in each of the Ohio, Frankfurt, and Seoul EC2 regions.  We chose to perform
measurements from multiple global vantage points to understand how DoH
performance varies not only by which resolver is used, but also which
geographic region the client is located in.  Each server utilized 8 GB of RAM
and 2 virtual CPU cores (the \texttt{t2.xlarge} instance type).
